\hiddensection{Шифрование с сохранением формата}\label{TEST.FMT}

В таблице~\ref{Table.TEST.FMTE} представлены примеры зашифрования 
с сохранением формата. В таблице символы слов $X,Y\in\ZZ_m^n$ 
представляются десятичными числами и разделяются запятыми. 

\begin{table}[H]
\caption{Зашифрование с сохранением формата}\label{Table.TEST.FMTE}
\begin{tabular}{|l|l|}
\hline
$K$ & 
$\hex{E9DEE72C~8F0C0FA6~2DDB49F4~6F739647~06075316~ED247A37~39CBA383~03A98BF6}$\\
\hline
$S$ & 
$\hex{BE329713~43FC9A48~A02A885F~194B09A1}$\\
\ddhline
$m$ & 10\\
\hline
$n$ & 10\\
\hline
$X$ &
$0,1,2,3,4,5,6,7,8,9$\\
\dhline
$Y$ & 
$6,9,3,4,7,7,0,3,5,2$\\
\ddhline
$m$ & 58\\
\hline
$n$ & 21\\
\hline
$X$ &
$0,1,2,3,4,5,6,7,8,9,10,11,12,13,14,15,16,17,18,19,20$\\
\dhline
$Y$ & 
$7,4,6,21,49,55,24,23,22,50,27,39,24,24,17,32,57,43,26,5,29$\\
\ddhline
$m$ & 65536\\
\hline
$n$ & 17\\
\hline
$X$ &
$0,1,2,3,4,5,6,7,8,9,10,11,12,13,14,15,16$\\
\dhline
$Y$ & 
$14290,31359,58054,51842,44653,34762,28652,48929,6541,13788,7784,46182,61098,$\\
& $43056,3564,21568,63878,$\\
\hline
\end{tabular}
\end{table}
