\section{Выработка имитовставки}\label{MAC}

\subsection{Интерфейс}\label{MAC.IFace}

Выработка имитовставки задается алгоритмом~$\algname{belt-mac}$.

Входными данными~$\algname{belt-mac}$ являются
сообщение~$X\in\{0,1\}^*$ и ключ~$K\in\{0,1\}^{256}$.
%
Выходными данными является имитовставка~$T\in\{0,1\}^{64}$.

Используется алгоритм~$\algname{belt-block}$, 
определенный в~\ref{BLOCK}.

\subsection{Вспомогательные преобразования и переменные}\label{MAC.Aux}

\paragraph{Преобразования $\varphi_1$ и $\varphi_2$.}
Преобразования $\varphi_1,\varphi_2\colon\{0,1\}^{128}\to\{0,1\}^{128}$
действуют на слово 
$u=u_1\parallel u_2\parallel u_3\parallel u_4$, 
$u_i\in\{0,1\}^{32}$, по правилам:
\begin{align*}
\varphi_1(u)&=u_2\parallel u_3\parallel u_4\parallel(u_1\oplus u_2),\\
\varphi_2(u)&=(u_1\oplus u_4)\parallel u_1\parallel u_2\parallel u_3.
\end{align*}

\paragraph{Отображение~$\psi$.}
Отображение $\psi$ ставит в соответствие двоичному слову $u$,
длина которого меньше $128$, слово
$
\psi(u)=u\parallel 1\parallel 0^{127-|u|}
$
длины~$128$.

\paragraph{Переменные.}
Используются переменные~$r,s\in\{0,1\}^{128}$.

\subsection{Алгоритм выработки имитовставки}\label{MAC.Alg}

Выработка имитовставки $\algname{belt-mac}(X,K)$ выполняется следующим образом:
\begin{enumerate}
\item
Определить $(X_1,X_2,\ldots,X_n)=\Split(X,128)$.
\item
Установить
$s\leftarrow 0^{128}$, $r\leftarrow \algname{belt-block}(s,K)$.
\item
Для $i=1,2,\ldots,n-1$ выполнить:
$s\leftarrow \algname{belt-block}(s\oplus X_i,K)$.
\item
Если $|X_n|=128$, то 
$s\leftarrow s\oplus X_n\oplus \varphi_1(r)$.
\item
Иначе, если $|X_n|<128$, то
$s\leftarrow s\oplus\psi(X_n)\oplus\varphi_2(r)$.
\item
Установить
$T\leftarrow\Lo(\algname{belt-block}(s,K), 64)$.
\item
Возвратить~$T$.
\end{enumerate}

\vskip9pt
\begin{note}
Примечание~--- Если $X=\perp$, то $n=1$ и $X_1=\perp$.
\end{note}
