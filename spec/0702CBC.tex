\section{Шифрование в режиме сцепления блоков}\label{CBC}

\subsection{Интерфейс}\label{CBC.IFace}

Шифрование в режиме сцепления блоков задается алгоритмами 
зашифрования~$\algname{belt-cbc}$ и расшифрования~$\algname{belt-cbc}^{-1}$.

Входными данными $\algname{belt-cbc}$ являются сообщение~$X\in\{0,1\}^*$, 
ключ~$K\in\{0,1\}^{256}$ и синхропосылка~$S\in\{0,1\}^{128}$. 
%
Длина $X$ должна быть не меньше $128$.
%
Выходными данными является зашифрованное сообщение~$Y\in\{0,1\}^{|X|}$.

Входными данными $\algname{belt-cbc}^{-1}$ являются зашифрованное 
сообщение~$Y\in\{0,1\}^*$, ключ~$K\in\{0,1\}^{256}$ и 
синхропосылка~$S\in\{0,1\}^{128}$.  
%
Длина $Y$ должна быть не меньше $128$.
%
Выходными данными является расшифрованное сообщение~$X\in\{0,1\}^{|Y|}$.

Используются алгоритмы~$\algname{belt-block}$ и~$\algname{belt-block}^{-1}$, 
определенные в~\ref{BLOCK}.

\subsection{Переменные}

Пусть $m=|X|\bmod 128$ при зашифровании 
и $m=|Y|\bmod 128$ при расшифровании.

Если $m\neq 0$, то используется переменная~$r\in\{0,1\}^{128-m}$.

\subsection{Алгоритм зашифрования}

Зашифрование~$\algname{belt-cbc}(X,K,S)$ выполняется следующим образом: 
\begin{enumerate}
\item
Определить $(X_1,X_2,\ldots,X_n)=\Split(X,128)$.
\item
Обозначить~$Y_0=S$.
\item
Если $|X_n|=128$, то
\begin{enumerate}
\item
для $i=1,2,\ldots,n$ выполнить: 
$Y_i\leftarrow \algname{belt-block}(X_i\oplus Y_{i-1},K)$.
\end{enumerate}
\item
Иначе, если $|X_n|<128$, то
\begin{enumerate}
\item
для $i=1,2,\ldots,n-2$ выполнить: 
$Y_i\leftarrow \algname{belt-block}(X_i\oplus Y_{i-1},K)$;
\item
$(Y_n\parallel r)\leftarrow \algname{belt-block}(X_{n-1}\oplus Y_{n-2},K)$;
\item
$Y_{n-1}\leftarrow \algname{belt-block}((X_n\oplus Y_n)\parallel r,K)$.
\end{enumerate}
\item
Возвратить
$Y=Y_1\parallel Y_2\parallel\ldots\parallel Y_n$.
\end{enumerate}

\subsection{Алгоритм расшифрования}

Расшифрование~$\algname{belt-cbc}^{-1}(Y,K,S)$ выполняется следующим образом: 
\begin{enumerate}
\item
Определить $(Y_1,Y_2,\ldots,Y_n)=\Split(Y,128)$.
\item
Обозначить~$Y_0=S$.
\item
Если $|Y_n|=128$, то
\begin{enumerate}
\item
для $i=1,2,\ldots,n$ выполнить: 
$X_i\leftarrow \algname{belt-block}^{-1}(Y_i,K)\oplus Y_{i-1}$.
\end{enumerate}
\item
Иначе, если $|Y_n|<128$, то
\begin{enumerate}
\item
для $i=1,2,\ldots,n-2$ выполнить: 
$X_i\leftarrow \algname{belt-block}^{-1}(Y_i,K)\oplus Y_{i-1}$;
\item
$(X_n\parallel r)\leftarrow \algname{belt-block}^{-1}(Y_{n-1},K)\oplus 
(Y_n\parallel 0^{128-m})$;
\item
$X_{n-1}\leftarrow \algname{belt-block}^{-1}(Y_n\parallel r,K)\oplus Y_{n-2}$.
\end{enumerate}
\item
Возвратить
$X=X_1\parallel X_2\parallel\ldots\parallel X_n$.
\end{enumerate}
