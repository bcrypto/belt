\section{Аутентифицированное шифрование данных}\label{AE}

\subsection{Интерфейс}\label{AE.IFace}

Аутентифицированное шифрование данных задается алгоритмами установки 
и снятия защиты. Первая схема аутентифицированного шифрования
представлена алгоритмами $\algname{belt-dwp}$ и~$\algname{belt-dwp}^{-1}$,
вторая~--- алгоритмами $\algname{belt-che}$ и~$\algname{belt-che}^{-1}$. 

Входными данными алгоритма установки защиты ($\algname{belt-dwp}$ или
$\algname{belt-che}$) являются сообщение~$X\in\{0,1\}^*$, ассоциированные
данные~$I\in\{0,1\}^*$, ключ~$K\in\{0,1\}^{256}$ и 
синхропосылка~$S\in\{0,1\}^{128}$. Длины~$X$ и~$I$ должны быть меньше~$2^{64}$. 
%
Выходными данными являются зашифрованное сообщение~$Y\in\{0,1\}^{|X|}$
и имитовставка~$T\in\{0,1\}^{64}$.

Входными данными алгоритма снятия защиты ($\algname{belt-dwp}^{-1}$ или
$\algname{belt-che}^{-1}$) являются зашифрованное
сообщение~$Y\in\{0,1\}^*$, ассоциированные данные~$I\in\{0,1\}^*$,
имитовставка~$T\in\{0,1\}^{64}$, ключ~$K\in\{0,1\}^{256}$ и синхропосылка
$S\in\{0,1\}^{128}$.
%
Длины $Y$ и $I$ должны быть меньше $2^{64}$.
%
Выходными данными является либо признак ошибки~$\perp$, либо расшифрованное 
сообщение~$X\in\{0,1\}^{|Y|}$. 
%
Возврат~$\perp$ означает нарушение целостности входных данных.

Используется алгоритм~$\algname{belt-block}$, 
определенный в~\ref{BLOCK}.

\subsection{Переменные и константы}\label{AE.Vars}

Используются переменные~$s,t\in\{0,1\}^{128}$.

В алгоритмах схемы~1 дополнительно используется переменная~$r\in\{0,1\}^{128}$. 

В алгоритмах схемы~2 используется слово~$C=\hex{02}\parallel 0^{120}$.
Слово представляет многочлен~$C(x)=x$ (см.~\ref{DEFS.Poly}).

\subsection{Алгоритм установки защиты (схема 1)}\label{AE.DWP.Wrap}

Установка защиты~$\algname{belt-dwp}(X,I,K,S)$ выполняется следующим образом:
\begin{enumerate}
\item
Определить 
\begin{enumerate}
\item
$(X_1,X_2,\ldots,X_n)=\Split(X, 128)$;
\item
$(I_1,I_2,\ldots,I_m)=\Split(I, 128)$. 
\end{enumerate}
\item
Установить
\begin{enumerate}
\item
$s\leftarrow\algname{belt-block}(S,K)$;
\item
$r\leftarrow \algname{belt-block}(s,K)$;
\item
$t\leftarrow\hex{B194BAC80A08F53B366D008E584A5DE4}$
(см. строку 1 таблицы~\ref{Table.SBOX}).
\end{enumerate}

\item
Для $i=1,2,\ldots,m$ выполнить:
\begin{enumerate}
\item
$t\leftarrow t\oplus (I_i\parallel 0^{128-|I_i|})$;
\item
$t\leftarrow t\ast r$.
\end{enumerate}

\item
Для $i=1,2,\ldots,n$ выполнить:
\begin{enumerate}
\item
$s\leftarrow s\boxplus\itob{1}_{128}$;
\item
$Y_i\leftarrow X_i\oplus\Lo(\algname{belt-block}(s,K), |X_i|)$;
\item
$t\leftarrow t\oplus (Y_i\parallel 0^{128-|Y_i|})$;
\item
$t\leftarrow t\ast r$.
\end{enumerate}

\item
Установить
$t\leftarrow t\oplus 
(\itob{|I|}_{64}\parallel\itob{|X|}_{64})$.
\item
Установить
$t\leftarrow \algname{belt-block}(t\ast r,K)$.
\item
Установить
$T\leftarrow \Lo(t, 64)$.
\item
Возвратить~$(Y,T)$, 
где~$Y=Y_1\parallel Y_2\parallel\ldots\parallel Y_n$.
\end{enumerate}

\subsection{Алгоритм снятия защиты (схема 1)}\label{AE.DWP.Unwrap}

Снятие защиты~$\algname{belt-dwp}^{-1}(Y,I,T,K,S)$ выполняется следующим образом:
\begin{enumerate}
\item
Определить 
\begin{enumerate}
\item
$(Y_1,Y_2,\ldots,Y_n)=\Split(Y, 128)$;
\item
$(I_1,I_2,\ldots,I_m)=\Split(I, 128)$. 
\end{enumerate}
\item
Установить
\begin{enumerate}
\item
$s\leftarrow\algname{belt-block}(S,K)$;
\item
$r\leftarrow \algname{belt-block}(s,K)$;
\item
$t\leftarrow\hex{B194BAC80A08F53B366D008E584A5DE4}$.
\end{enumerate}

\item
Для $i=1,2,\ldots,m$ выполнить:
\begin{enumerate}
\item
$t\leftarrow t\oplus (I_i\parallel 0^{128-|I_i|})$;
\item
$t\leftarrow t\ast r$.
\end{enumerate}

\item\label{Step.AE.DWP.StepA}
Для $i=1,2,\ldots,n$ выполнить:
\begin{enumerate}
\item
$t\leftarrow t\oplus (Y_i\parallel 0^{128-|Y_i|})$;
\item
$t\leftarrow t\ast r$.
\end{enumerate}

\item
Установить
$t\leftarrow t\oplus 
(\itob{|I|}_{64}\parallel\itob{|Y|}_{64})$.

\item
Установить
$t\leftarrow \algname{belt-block}(t\ast r,K)$.

\item\label{Step.AE.DWP.VerifyMAC}
Если $T\neq \Lo(t, 64)$, то возвратить~$\perp$.

\item\label{Step.AE.DWP.StepD}
Для $i=1,2,\ldots,n$ выполнить:
\begin{enumerate}
\item
$s\leftarrow s\boxplus\itob{1}_{128}$;
\item
$X_i\leftarrow Y_i\oplus \Lo(\algname{belt-block}(s,K),|Y_i|)$.
\end{enumerate}

\item
Возвратить~$X=X_1\parallel X_2\parallel\ldots\parallel X_n$.
\end{enumerate}

\subsection{Алгоритм установки защиты (схема 2)}\label{AE.CHE.Wrap}

Установка защиты~$\algname{belt-che}(X,I,K,S)$ выполняется следующим образом:
\begin{enumerate}
\item
Определить 
\begin{enumerate}
\item
$(X_1,X_2,\ldots,X_n)=\Split(X, 128)$;
\item
$(I_1,I_2,\ldots,I_m)=\Split(I, 128)$. 
\end{enumerate}
\item
Установить
\begin{enumerate}
\item
$s\leftarrow\algname{belt-block}(S,K)$;
\item
$r\leftarrow s$;
\item
$t\leftarrow\hex{B194BAC80A08F53B366D008E584A5DE4}$.
\end{enumerate}

\item
Для $i=1,2,\ldots,m$ выполнить:
\begin{enumerate}
\item
$t\leftarrow t\oplus (I_i\parallel 0^{128-|I_i|})$;
\item
$t\leftarrow t\ast r$.
\end{enumerate}

\item
Для $i=1,2,\ldots,n$ выполнить:
\begin{enumerate}
\item
$s\leftarrow (s\ast C)\oplus\itob{1}_{128}$;
\item
$Y_i\leftarrow X_i\oplus\Lo(\algname{belt-block}(s,K), |X_i|)$;
\item
$t\leftarrow t\oplus (Y_i\parallel 0^{128-|Y_i|})$;
\item
$t\leftarrow t\ast r$.
\end{enumerate}

\item
Установить
$t\leftarrow t\oplus 
(\itob{|I|}_{64}\parallel\itob{|X|}_{64})$.
\item
Установить
$t\leftarrow \algname{belt-block}(t\ast r,K)$.
\item
Установить
$T\leftarrow \Lo(t, 64)$.
\item
Возвратить~$(Y,T)$, 
где~$Y=Y_1\parallel Y_2\parallel\ldots\parallel Y_n$.
\end{enumerate}

\subsection{Алгоритм снятия защиты (схема 2)}\label{AE.CHE.Unwrap}

Снятие защиты~$\algname{belt-che}^{-1}(Y,I,T,K,S)$ выполняется следующим образом:
\begin{enumerate}
\item
Определить 
\begin{enumerate}
\item
$(Y_1,Y_2,\ldots,Y_n)=\Split(Y, 128)$;
\item
$(I_1,I_2,\ldots,I_m)=\Split(I, 128)$. 
\end{enumerate}
\item
Установить
\begin{enumerate}
\item
$s\leftarrow\algname{belt-block}(S,K)$;
\item
$r\leftarrow s$;
\item
$t\leftarrow\hex{B194BAC80A08F53B366D008E584A5DE4}$.
\end{enumerate}

\item
Для $i=1,2,\ldots,m$ выполнить:
\begin{enumerate}
\item
$t\leftarrow t\oplus (I_i\parallel 0^{128-|I_i|})$;
\item
$t\leftarrow t\ast r$.
\end{enumerate}

\item\label{Step.AE.CHE.StepA}
Для $i=1,2,\ldots,n$ выполнить:
\begin{enumerate}
\item
$t\leftarrow t\oplus (Y_i\parallel 0^{128-|Y_i|})$;
\item
$t\leftarrow t\ast r$.
\end{enumerate}

\item
Установить
$t\leftarrow t\oplus (\itob{|I|}_{64}\parallel\itob{|Y|}_{64})$.

\item
Установить
$t\leftarrow\algname{belt-block}(t\ast r,K)$.

\item\label{Step.AE.CHE.VerifyMAC}
Если $T\neq\Lo(t, 64)$, то возвратить~$\perp$.

\item\label{Step.AE.CHE.StepD}
Для $i=1,2,\ldots,n$ выполнить:
\begin{enumerate}
\item
$s\leftarrow (s\ast C)\oplus\itob{1}_{128}$;
\item
$X_i\leftarrow Y_i\oplus \Lo(\algname{belt-block}(s,K), |Y_i|)$.
\end{enumerate}
\item
Возвратить~$X=X_1\parallel X_2\parallel\ldots\parallel X_n$.
\end{enumerate}

\begin{note}
Примечание~1~--- Если длина имитовставки, вычисляемой при установке защиты, 
сокращается (см.~\ref{COMMON.MAC}), то на вход алгоритма снятия 
защиты должна подаваться сокращенная имитовставка, а проверка~$T\neq\Lo(t,64)$ 
на шаге~\ref{Step.AE.DWP.VerifyMAC} этого алгоритма должна быть изменена на 
проверку~$T\neq\Lo(t,|T|)$.
\end{note}

\begin{note}
Примечание~2~--- В алгоритме снятия защиты расшифрование на 
шаге~\ref{Step.AE.DWP.StepD} может выполняться одновременно 
с вычислением имитовставки на шаге~\ref{Step.AE.DWP.StepA}. 
Однако расшифрование может оказаться бесполезным, если на 
шаге~\ref{Step.AE.DWP.VerifyMAC} имитовставка окажется некорректной.
\end{note}

\begin{note}
Примечание~3~--- При вычислении $(Y,T)=\algname{belt-dwp}(X,I,K,S)$ 
разрешается выдавать один или несколько промежуточных результатов 
$(Y',T')=\algname{belt-dwp}(X',I,K,S)$. 
%
Здесь~$X'$~--- префикс~$X$, а~$Y'$ будет префиксом~$Y$.
%
Выдача промежуточных результатов позволяет дискретизировать процесс защиты, 
что оказывается полезным при передаче сообщений~$X$ большой длины.
%
При снятии защиты промежуточная пара~$(Y',T')$ обрабатывается с 
помощью алгоритма~$\algname{belt-dwp}^{-1}$ обычным образом. 
%
Если $\algname{belt-dwp}^{-1}(Y',I,T',K,S)=\perp$, то снятие 
защиты должно быть прервано.
%
Сказанное относится также к алгоритмам $\algname{belt-che}$ и $\algname{belt-che}^{-1}$.
\end{note}


