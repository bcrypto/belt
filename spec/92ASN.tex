\begin{appendix}{Б}{рекомендуемое}{Модуль АСН.1}\label{ASN}

\mbox{}

В модуле АСН.1 алгоритмам настоящего стандарта назначаются идентификаторы.
Назначение выполняется указанием имени алгоритма (группы алгоритмов) и 
соответствующего идентификатора. Имя представляет собой объединение 
короткого имени алгоритма, указанного в интерфейсе, с 3-символьным 
суффиксом~\algname{128}, \algname{192} или \algname{256}. Суффикс обозначает 
длину используемого ключа. Суффикс не добавляется к именам 
\algname{belt-compress}, \algname{belt-hash}, \algname{belt-keyexpand}, 
\algname{belt-keyrep}.

Группа может включать два алгоритма: зашифрования (установки защиты)
и расшифрования (снятия защиты). Идентификатор относится к обоим 
алгоритмам группы.

В модуле АСН.1 дополнительно определяются форматы следующих параметров:
{\tabcolsep 0pt
\begin{longtable}{lrp{14cm}}
\texttt{IV} &\mbox{~~~~~}&
Синхропосылка в алгоритмах
\algname{belt-cbcXXX},
\algname{belt-cfbXXX},
\algname{belt-ctrXXX},
\algname{belt-dwpXXX}, 
\algname{belt-cheXXX}, 
\algname{belt-bdeXXX},
\algname{belt-sdeXXX},
\algname{belt-fmtXXX};\\
%
\texttt{KeyHeader} &&
Заголовок ключа в алгоритмах 
\algname{belt-kwpXXX},
\algname{belt-keyrep};\\
%
\texttt{KeyLevel} &&
Уровень ключа в алгоритме
\algname{belt-keyrep}.
\end{longtable}
}

Если алгоритм шифрования описывается типом
\begin{verbatim}
AlgorithmIdentifier ::= SEQUENCE {
  algorithm OBJECT IDENTIFIER,
  parameters ANY DEFINED BY algorithm OPTIONAL
}   
\end{verbatim}
то в компоненте~\texttt{algorithm} должен указываться идентификатор
алгоритма, а в компоненте~\texttt{parameters}~--- используемая 
синхропосылка типа~\texttt{IV}.
%
Если синхропосылка не используется, то~компонент~\texttt{parameters}
должен быть опущен.
%
Отсутствие~\texttt{parameters} при описании алгоритмов~\algname{belt-fmtXXX}
означает, что используется нулевая синхропосылка.

Модуль АСН.1 имеет следующий вид:

\verbatiminput{belt-module-v2.asn}

\end{appendix}
