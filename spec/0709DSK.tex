\section{Дисковое шифрование}\label{DSK}

\subsection{Интерфейс}\label{DSK.IFace}

Блоковое дисковое шифрование задается алгоритмами 
зашифрования~$\algname{belt-bde}$ и расшифрования~$\algname{belt-bde}^{-1}$.
%
Секторное дисковое шифрование задается алгоритмами 
зашифрования~$\algname{belt-sde}$ и расшифрования~$\algname{belt-sde}^{-1}$.

Входными данными~$\algname{belt-bde}$, $\algname{belt-sde}$ являются 
сообщение~$X\in\{0,1\}^{128*}$, ключ~$K\in\{0,1\}^{256}$ и 
синхропосылка~$S\in\{0,1\}^{128}$. 
%
При блоковом шифровании длина~$X$ должна быть не меньше~$128$, при 
секторном~--- не меньше~$256$.
%
Выходными данными является зашифрованное сообщение~$Y\in\{0,1\}^{|X|}$.

Входными данными~$\algname{belt-bde}^{-1}$, $\algname{belt-sde}^{-1}$ являются 
зашифрованное сообщение~$Y\in\{0,1\}^{128*}$, ключ~$K\in\{0,1\}^{256}$ и 
синхропосылка~$S\in\{0,1\}^{128}$. 
%
При блоковом шифровании длина~$Y$ должна быть не меньше~$128$, при 
секторном~--- не меньше~$256$.
%
Выходными данными является расшифрованное сообщение~$X\in\{0,1\}^{|Y|}$.

Слова~$X$ и~$Y$ представляют собой содержимое дискового сектора до и после
зашифрования. Синхропосылка~$S$ строится по номеру сектора.
Например, $S=\itob{D}_{128}$ для сектора номер $D$.

Используются алгоритмы~$\algname{belt-block}$ и~$\algname{belt-block}^{-1}$, 
определенные в~\ref{BLOCK}. 
%
При секторном шифровании дополнительно используются 
алгоритмы~$\algname{belt-wblock}$ и~$\algname{belt-wblock}^{-1}$, 
определенные в~\ref{WBLOCK}.  

\subsection{Переменные и константы}\label{DSK.Vars}

Используется переменная~$s\in\{0,1\}^{128}$. 

В алгоритмах~$\algname{belt-bde}$, $\algname{belt-bde}^{-1}$ 
используется слово~$C=\hex{02}\parallel 0^{120}$.
Слово представляет многочлен~$C(x)=x$ (см.~\ref{DEFS.Poly}).

\subsection{Алгоритм блокового зашифрования}\label{DSK.BDE.Encr}

Зашифрование~$\algname{belt-bde}(X,K,S)$ выполняется следующим образом: 
\begin{enumerate}
\item
Определить~$(X_1,X_2,\ldots,X_n)=\Split(X, 128)$.
\item
Установить~$s\leftarrow \algname{belt-block}(S,K)$.
\item
Для~$i=1,2,\ldots,n$ выполнить: 
\begin{enumerate}
\item
$s\leftarrow s \ast C$;
\item
$Y_i\leftarrow\algname{belt-block}(X_i\oplus s, K)\oplus s$.
\end{enumerate}
\item
Возвратить~$Y=Y_1\parallel Y_2\parallel\ldots\parallel Y_n$.
\end{enumerate}

\subsection{Алгоритм блокового расшифрования}\label{DSK.BDE.Decr}

Расшифрование~$\algname{belt-bde}^{-1}(Y,K,S)$ выполняется следующим образом: 
\begin{enumerate}
\item
Определить~$(Y_1,Y_2,\ldots,Y_n)=\Split(Y,128)$.
\item
Установить~$s\leftarrow\algname{belt-block}(S,K)$.
\item
Для~$i=1,2,\ldots,n$ выполнить: 
\begin{enumerate}
\item
$s\leftarrow s \ast C$;
\item
$X_i\leftarrow\algname{belt-block}^{-1}(Y_i\oplus s, K)\oplus s$.
\end{enumerate}
\item
Возвратить~$X=X_1\parallel X_2\parallel\ldots\parallel X_n$.
\end{enumerate}

\subsection{Алгоритм секторного зашифрования}\label{DSK.SDE.Encr}

Зашифрование~$\algname{belt-sde}(X,K,S)$ выполняется следующим образом: 
\begin{enumerate}
\item
Установить~$Y\leftarrow X$.
\item
Записать~$Y=Y_1\parallel Y_2$, где $|Y_1|=128$.
\item
Установить~$s\leftarrow \algname{belt-block}(S,K)$.
\item
Установить~$Y_1\leftarrow Y_1\oplus s$.
\item
Установить~$Y\leftarrow\algname{belt-wblock}(Y,K)$.
\item
Установить~$Y_1\leftarrow Y_1\oplus s$.
\item
Возвратить~$Y$.
\end{enumerate}

\subsection{Алгоритм секторного расшифрования}\label{DSK.SDE.Decr}

Расшифрование~$\algname{belt-sde}^{-1}(Y,K,S)$ выполняется следующим образом: 
\begin{enumerate}
\item
Установить~$X\leftarrow Y$.
\item
Записать~$X=X_1\parallel X_2$, где $|X_1|=128$.
\item
Установить~$s\leftarrow \algname{belt-block}(S,K)$.
\item
Установить~$X_1\leftarrow X_1\oplus s$.
\item
Установить~$X\leftarrow\algname{belt-wblock}^{-1}(X,K)$.
\item
Установить~$X_1\leftarrow X_1\oplus s$.
\item
Возвратить~$X$.
\end{enumerate}
