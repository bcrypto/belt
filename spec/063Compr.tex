\section{Cжатие}\label{COMPR}

\subsection{Интерфейс}\label{COMPR.IFace}

Cжатие задается алгоритмом~$\algname{belt-compress}$.

Входными данными $\algname{belt-compress}$ является слово~$X\in\{0,1\}^{512}$.
%
Выходными данными являются слова~$S\in\{0,1\}^{128}$ 
и~$Y\in\{0,1\}^{256}$. 
%
Слово~$S$ является промежуточным результатом сжатия~$X$,
этот выход может игнорироваться.
%
Слово~$Y$ является окончательным результатом сжатия.

Используется алгоритм~$\algname{belt-block}$, определенный в~\ref{BLOCK}.

\subsection{Алгоритм сжатия}\label{COMPR.Alg}

Сжатие~$\algname{belt-compress}(X)$ выполняется следующим образом:
%
\begin{enumerate}
\item
Определить $(X_1,X_2,X_3,X_4)=\Split(X,128)$.
\item
Установить
$S\leftarrow\algname{belt-block}(X_3\oplus X_4,X_1\parallel X_2)\oplus X_3\oplus X_4$.
\item
Установить
$Y_1\leftarrow\algname{belt-block}(X_1,S\parallel X_4)\oplus X_1$.
\item
Установить
$Y_2\leftarrow\algname{belt-block}(X_2,(S\oplus 1^{128})\parallel X_3)\oplus X_2$.
\item
Возвратить~$(S,Y)$, где $Y=Y_1\parallel Y_2$.
\end{enumerate}

