\thispagestyle{empty}

\noindent
{\bf ГОСУДАРСТВЕННЫЙ СТАНДАРТ} \hfill {\bf\draftlogo}\\
\noindent
{\bf РЕСПУБЛИКИ~БЕЛАРУСЬ}\\[-9pt]
\hrule height 1pt
\vskip0.4mm
\hrule height 2pt

\vskip2cm
\noindent
{\bf\Large Информационные технологии и безопасность}\\[10pt]
{\bf\large АЛГОРИТМЫ ШИФРОВАНИЯ И КОНТРОЛЯ}\\
{\bf\large ЦЕЛОСТНОСТИ}

\vskip2cm
\noindent
{\bf\Large Iнфармацыйныя тэхналогii i бяспека}\\[10pt]
{\bf\large АЛГАРЫТМЫ ШЫФРАВАННЯ I КАНТРОЛЮ}\\
{\bf\large ЦЭЛАСНАСЦI}

\vskip9cm
\hrule height 1pt
\vskip0.4mm
\hrule height 2pt
\noindent
\begin{tabular}{p{5cm}cp{4cm}}
\vtop{\null\hbox{{\includegraphics[width=2.6cm]{../figs/stb}}}} & \hspace{6cm} & 
\mbox{}\newline\mbox{}\newline\newline Госстандарт\newline Минск\\
\end{tabular}

\pagebreak

\hrule
\vskip2mm

УДК~004.056.55(083.74)(476)\hfill
МКС~35.240.40\hfill
\mbox{}

\vskip0.5mm
 
{\bf Ключевые слова}: криптографический алгоритм,
шифрование, имитозащита, аутентифицированное шифрование,
хэширование, управление ключами  

\vskip0.5mm

\hrule 

\rule{0pt}{5mm}
 
\centerline{\bf Предисловие} 

Цели, основные принципы, положения по государственному регулированию и 
управлению в области технического нормирования и стандартизации 
установлены Законом Республики Беларусь <<О техническом нормировании и 
стандартизации>>.  

1~РАЗРАБОТАН учреждением Белорусского государственного университета 
<<Науч\-но-исследовательский институт прикладных проблем математики и 
информатики>> 

ВНЕСЕН Оперативно-аналитическим центром при Президенте Республики Беларусь 

2~УТВЕРЖДЕН И ВВЕДЕН В ДЕЙСТВИЕ постановлением Госстандарта Республики 
Беларусь от 1 октября 2020 г. \No~56

3~ВЗАМЕН СТБ 34.101.31-2011 

\vfill

\hrule
\vskip1mm
Издан на русском языке

\pagebreak
