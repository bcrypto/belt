\chapter{Термины и определения}

В настоящем стандарте применяются  
следующие термины с соответствующими определениями:

{\bf \thedefctr~аутентифицированное шифрование}:
Одновременные шифрование и имитозащита.

{\bf \thedefctr~блок}:
Двоичное слово длины~$128$.

\begin{note}
Примечание~--- при разбиении двоичного слова на блоки последний блок
может быть неполным.
\end{note}

{\bf \thedefctr~заголовок ключа}:
Блок, содержащий открытые атрибуты ключа.

{\bf \thedefctr~зашифрование}:
Преобразование сообщения,
направленное на обеспечение его конфиденциальности,
которое выполняется с использованием ключа.

{\bf \thedefctr~имитовставка}:
Двоичное слово, 
которое определяется по сообщению с использованием ключа 
и служит для контроля целостности и подлинности сообщения.

{\bf \thedefctr~имитозащита}:
Контроль целостности и подлинности сообщений, 
который реализуется путем выработки и проверки имитовставок.

{\bf \thedefctr~ключ (секретный)}:
Параметр, который управляет операциями шифрования 
и имитозащиты и который известен только определенным сторонам.

{\bf \thedefctr~конфиденциальность}:
Гарантия того, что сообщения доступны для понимания или использования
только тем сторонам, которым они предназначены.

{\bf \thedefctr~октет}:
Двоичное слово длины~$8$.

{\bf \thedefctr~подлинность}:
Гарантия того, что сторона действительно является
владельцем (создателем, отправителем) определенного сообщения.

{\bf \thedefctr~преобразование ключа}:
Построение по исходному ключу набора новых ключей 
с различными заголовками.

{\bf \thedefctr~расширение ключа}:
Дополнение ключа новыми символами до получения ключа определенной длины.

{\bf \thedefctr~расшифрование}:
Преобразование, обратное зашифрованию.

{\bf \thedefctr~синхропосылка}:
Открытые входные данные криптографического алгоритма,
которые обеспечивают уникальность результатов 
криптографического преобразования на фиксированном ключе.

{\bf \thedefctr~снятие защиты}:
Проверка имитовставок и расшифрование.

{\bf \thedefctr~сообщение}:
Двоичное слово конечной длины.

{\bf \thedefctr~установка защиты}:
Зашифрование и вычисление имитовставок.

{\bf \thedefctr~хэш-значение}:
Двоичное слово фиксированной длины, 
которое определяется по сообщению без использования ключа и 
служит для контроля целостности сообщения и для представления 
сообщения в (необратимо) сжатой форме.

{\bf \thedefctr~хэширование}:
Выработка хэш-значений.

{\bf \thedefctr~целостность}:
Гарантия того, что в сообщение не внесены изменения
при его хранении, передаче и обработке.

{\bf \thedefctr~шифрование}:
Зашифрование или расшифрование.

