\chapter{Обозначения}\label{DEFS}

\section{Список обозначений}\label{DEFS.List}

{\tabcolsep 0pt
\begin{longtable}{lrp{13.2cm}}
$\perp$  & \hspace{2mm} &
специальный объект или ситуация: 
пустое слово, игнорируемая переменная, ошибка;
\\[4pt]
%
$\Sigma^n$  &&
множество всех слов длины~$n$ в алфавите~$\Sigma$;
\\[4pt]
%
$\Sigma^*$ &&
множество всех слов конечной длины в алфавите~$\Sigma$
(включая пустое слово~$\perp$ длины~$0$);
\\[4pt]
$|u|$ &&
длина слова~$u\in\Sigma^*$;
\\[4pt]
%
$\Sigma^{n*}$ &&
множество всех слов из~$\Sigma^*$, длина которых кратна~$n$;
\\[4pt]
%
$\alpha^n$ &&
для~$\alpha\in\Sigma$ слово из~$n$ экземпляров~$\alpha$;
\\[4pt]
%
$\Lo(u,m)$ &&
для~$u\in\Sigma^*$ и~$m\leq|u|$
слово из первых~$m$ символов~$u$;
\\[4pt]
%
$u\parallel v$ &&
для~$u=u_1 u_2\ldots u_n\in\Sigma^n$ 
и~$v=v_1 v_2\ldots v_m\in\Sigma^m$
слово~$u_1 u_2\ldots u_n v_1 v_2\ldots v_m$
(конкатенация);
\\[4pt]
%
$\Splitz(u,m)$ &&
для $u\in\Sigma^*$ и натурального $m$ представление~$u$
в виде набора $(u_1,u_2,\ldots,u_n)$ фрагментов~$u_i\in\Sigma^*$ 
таких, что
$u=u_1\parallel u_2\parallel\ldots\parallel u_n$,
$|u_1|=|u_2|=\ldots=|u_{n-1}|=m$ и $0<|u_n|\leq m$, 
причем набор пуст ($n=0$), если $u=\perp$;
\\[4pt]
%
$\Split(u,m)$ &&
$\Splitz(u,m)$, если $u\neq\perp$, и одноэлементный 
набор $(\perp)$ в противном случае;
\\[4pt]
%
$U\bmod m$ &&
для целого~$U$ и натурального~$m$ остаток от деления~$U$ на~$m$; 
\\[4pt]
%
$\ZZ_m$ && 
алфавит~$\{0,1,\ldots,m-1\}$, $m\geq 2$;
\\[4pt]
%
$u\oplus v$ &&
для~$u=u_1 u_2\ldots u_n\in\ZZ_m^n$ 
и~$v=v_1 v_2\ldots v_n\in\ZZ_m^n$
слово~$w=w_1 w_2\ldots w_n\in\ZZ_m^n$
из символов~$w_i=(u_i+v_i)\bmod{m}$;
\\[4pt]
%
$u\ominus v$ &&
для~$u,v\in\ZZ_m^n$ 
слово~$w\in\ZZ_m^n$ такое, что~$u=v\oplus w$
(при $m=2$ символы~$\oplus$ и~$\ominus$ эквивалентны);
\\[4pt]
%
$\btoi{u}_m$ && 
для~$u=u_1 u_2 \ldots u_n\in\ZZ_m^n$
число~$U=u_1 + m u_2 + \ldots + m^{n-1} u_n$;
\\[4pt]
%
$\btoi{u}$ && 
а)~для октета~$u=u_1 u_2\ldots u_8\in\{0,1\}^8$
число~$u_1 2^7+u_2 2^6 + \ldots + u_8$,\\[2pt]
%
&&
б)~для~$u=u_1\parallel u_2\parallel\ldots\parallel u_n$, $u_i\in\{0,1\}^8$,
число~$\Btoi{\btoi{u_1}\btoi{u_2}\ldots\btoi{u_n}}_{256}$;
\\[4pt]
%
$\itob{U}_{m,n}$ && 
для неотрицательного целого~$U$
и натуральных~$m$, $n$ слово~$u\in\ZZ_m^n$ такое, 
что~$\btoi{u}_m=U\bmod m^n$; 
\\[4pt]
%
$\itob{U}_{8n}$ && 
для неотрицательного целого~$U$ и натурального~$n$ 
слово~$u\in\{0,1\}^{8n}$ такое, что~$\btoi{u}=U\bmod 2^{8n}$; 
\\[4pt]
%
$\hex{01234\ldots}$ && 
представление $u\in\{0,1\}^{4*}$ шестнадцатеричным словом,
при котором последовательным четырем символам~$u$ соответствует
один шестнадцатеричный символ
(например, $10100010=\hex{A2}$);
\\[4pt]
%
$u\boxplus v$  &&
для~$u,v\in\{0,1\}^{8n}$ слово 
$\Itob{\btoi{u}+\btoi{v}}_{8n}$;
\\[4pt]
%
$u\boxminus v$ &&
для~$u,v\in\{0,1\}^{8n}$ 
слово $w\in\{0,1\}^{8n}$ такое, что $u=v\boxplus w$;
\\[4pt]
%
$\lfloor z\rfloor$ &&
для вещественного $z$ максимальное целое, не превосходящее~$z$;
\\[4pt]
%
$\lceil z\rceil$ &&
для вещественного $z$ минимальное целое, не меньшее~$z$;
\\[4pt]
%
$\ShLo(u)$ &&
для~$u\in\{0,1\}^{8n}$ 
слово~$\itob{\left\lfloor\btoi{u}/2\right\rfloor}_{8n}$;
\\[4pt]
%
$\ShHi(u)$ &&
для~$u\in\{0,1\}^{8n}$ 
слово~$\itob{2\btoi{u}}_{8n}$;
\\[4pt]
%
$\varphi^r(u)$ &&
для слова $u$ и преобразования $\varphi$
результат $r$-кратного действия $\varphi$ на $u$
(например, $\ShLo^r(u)$~--- результат $r$-кратного действия $\ShLo$);
\\[4pt]
%
$\RotHi(u)$ &&
для~$u\in\{0,1\}^{8n}$ 
слово $\ShHi(u)\oplus\ShLo^{8n-1}(u)$;
\\[4pt]
%
$\FF_2$ &&
поле из двух элементов $0$ и $1$;
\\[4pt]
%
$\FF_2[x]$ &&
кольцо многочленов над полем $\FF_2$;
\\[4pt]
%
$u(x)$ &&
а)~для октета~$u=u_1 u_2\ldots u_8\in\{0,1\}^8$
многочлен $u_1 x^7+u_2 x^6 + \ldots + u_8$,\\[2pt]
%
&&
б)~для~$u=u_1\parallel u_2\parallel\ldots\parallel u_n$, $u_i\in\{0,1\}^8$,
многочлен~$u_1(x)+x^8 u_2(x)+\ldots+x^{8(n-1)}u_n(x)$;
\\[4pt]
%
$u(x)\bmod f(x)$ &&
для $u(x)\in\FF_2[x]$ и ненулевого $f(x)\in\FF_2[x]$
остаток от деления $u(x)$ на $f(x)$;
\\[4pt]
%
$u\ast v$ &&
для $u,v\in\{0,1\}^{128}$ слово~$w\in\{0,1\}^{128}$ такое, 
что~$w(x)=u(x)v(x)\bmod x^{128}+x^7+x^2+x+1$;
\\[4pt]
%
$\algname{alg}(u_1,u_2,\ldots)$ &&
вызов алгоритма~\algname{alg} с входными данными~$u_1,u_2,\ldots$;
\\[4pt]
%
$a\leftarrow u$ &&
присвоение переменной $a$ значения $u$;
\\[4pt]
%
$a\leftrightarrow b$ &&
перестановка значений переменных $a$ и $b$;
\\[4pt]
%
$(a_1,a_2)\leftarrow(u_1,u_2)$ &&
присвоение переменной $a_1$ значения~$u_1$, переменной~$a_2$ 
значения~$u_2$;
\\[4pt]
%
$(\perp,a_2)\leftarrow(u_1,u_2)$ &&
то же самое, что~$a_2\leftarrow u_2$ (например, $u_1$ и~$u_2$~--- выходы 
алгоритма и выход~$u_1$ игнорируется).
\\[4pt]
\end{longtable}
} % tabcolsep
\setcounter{table}{0}

\section{Пояснения к обозначениям}

\subsection{Слова}\label{DEFS.Words}

Слово в алфавите~$\Sigma$ представляет собой последовательность символов
этого алфавита. Символы нумеруются слева направо от единицы.
Примеры алфавитов: 
$\ZZ_2=\{0,1\}$ (двоичный),
$\ZZ_{10}$ (десятичный), 
$\ZZ_{16}$ (шестнадцатеричный),
$\ZZ_{256}$ (байтовый).

Символы шестнадцатеричного алфавита (числа от~$0$ до~$15$) обозначаются 
знаками~$\hex{0},\hex{1},\ldots,\hex{F}$. При записи шестнадцатеричного
слова индекс~$16$ после всех символов, кроме последнего, исключается.

Слово~$u=u_1 u_2\ldots u_n$ в алфавите~$\ZZ_m$ является записью 
числа~$U=\btoi{u}_m$ в системе счисления по основанию~$m$. 
При этом первый символ слова является младшим, последний~--- старшим.
%
Таким образом, используется соглашение <<от младших к старшим>> 
(little-endian), распространенное для многих современных процессоров при 
$m=256$. 

\subsection{Двоичные слова}\label{DEFS.BinWords}

В настоящем подразделе в качестве примера рассматривается двоичное слово
$$
w=1011 0001 1001 0100 1011 1010 1100 1000.
$$
В этом слове первый символ~--- $1$, 
второй~--- $0$, \ldots, последний~--- $0$.

Двоичное слово разбивается на тетрады из четверок последовательных 
двоичных символов. 
%
Тетрады кодируются шестнадцатеричными символами по правилам, заданным в 
таблице~\ref{Table.Hex}.

\begin{table}[H]
\caption{}\label{Table.Hex}
\begin{tabular}{|c|c||c|c||c|c||c|c|}
\hline
тетрада & символ & тетрада & символ & тетрада & символ & тетрада & символ\\
\hline
0000 & $\hex{0}$ & 0001 & $\hex{1}$ & 
0010 & $\hex{2}$ & 0011 & $\hex{3}$\\
0100 & $\hex{4}$ & 0101 & $\hex{5}$ & 
0110 & $\hex{6}$ & 0111 & $\hex{7}$\\ 
1000 & $\hex{8}$ & 1001 & $\hex{9}$ & 
1010 & $\hex{A}$ & 1011 & $\hex{B}$\\ 
1100 & $\hex{C}$ & 1101 & $\hex{D}$ & 
1110 & $\hex{E}$ & 1111 & $\hex{F}$\\ 
\hline
\end{tabular}
\end{table}

Например, слово $w$ кодируется следующим образом:
$$
\hex{B194BAC8}.
$$

Пары тетрад образуют октеты. Последовательные октеты слова~$w$ имеют вид:
$$
1011 0001=\hex{B1},\ 
1001 0100=\hex{94},\ 
1011 1010=\hex{BA},\  
1100 1000=\hex{C8}.
$$

Октету $u=u_1 u_2\ldots u_8$ ставится в соответствие байт~--- 
число $\btoi{u}=2^7u_1+2^6 u_2+\ldots + u_8$. 
Например, октетам $w$ соответствуют байты
$$
177=2^7+2^5+2^4+1,\ 
148=2^7+2^4+2^2,\ 
186=2^7+2^5+2^4+2^3+2^1,\ 
200=2^7+2^6+2^3.
$$

Число ставится в соответствие не только октетам, но и любому другому
двоичному слову, длина которого кратна~$8$:
сначала строится слово из байтов, затем применяется функция~$\btoi{\cdot}_{256}$.
%
Например, 
$$
\btoi{w}=\btoi{177,148,186,200}_{256}=177+2^{8}\cdot 148+2^{16}\cdot 
186+2^{24}\cdot 200 = 3367670961. 
$$

При отождествлении слов с числами удобно представить себе 
гипотетический регистр, разрядность которого совпадает с длиной слова.
В самый правый октет регистра загружается первый октет слова, 
во второй справа октет регистра~--- второй октет слова и так далее,
пока, наконец, в самый левый октет регистра не загружается последний 
октет слова.
%
Например, для $w$ содержимое регистра имеет 
вид:
$$
\hex{C8BA94B1}=
11001000 10111010 10010100 10110001.
$$

При таком представлении операции $\ShLo$, $\ShHi$, $\RotHi$ состоят в
сдвигах содержимого регистра: 
$\ShLo$~--- вправо (в сторону младших разрядов),
$\ShHi$~--- влево (в сторону старших разрядов)
и~$\RotHi$~--- циклически влево,
причем при сдвигах $\ShLo$ и~$\ShHi$ в освободившиеся разряды 
регистров записываются нули.
%
Например, предыдущий регистр изменяется при сдвигах следующим образом:
$$      
\begin{array}{rl}
\ShLo: &\hex{645D4A58}=
0 1100100 0 1011101 0 1001010 0 1011000,\\
\ShHi: &\hex{91752962}= 
1001000 1 0111010 1 0010100 1 0110001 0,\\
\RotHi: &\hex{91752963}=
1001000 1 0111010 1 0010100 1 0110001 1.
\end{array}
$$
Выгружая из регистра октеты слева направо, 
получаем следующие результаты:
$$
\begin{array}{rl}
\ShLo(w)&=  \hex{584A5D64},\\
\ShHi(w)&=  \hex{62297591},\\
\RotHi(w)&= \hex{63297591}.
\end{array}
$$

Перестановки октетов при загрузке слова в регистр и при выгрузке из 
регистра в современных процессорах выполняются неявно.

При сдвигах на число позиций, кратное $8$, операции $\ShLo$, $\ShHi$, $\RotHi$
интерпретируются намного проще и состоят в сдвиге октетов исходного слова: 
при $\ShLo$~--- в сторону первых октетов, 
при $\ShHi$~--- в сторону последних октетов,
при $\RotHi$~--- циклически в сторону последних октетов.
Например,
$$
\begin{array}{rl}
\ShLo^8(w)&=  \hex{94BAC800},\\
\ShHi^8(w)&=  \hex{00B194BA},\\
\RotHi^8(w)&= \hex{C8B194BA}.
\end{array}
$$

\subsection{Двоичные слова как многочлены}\label{DEFS.Poly}

Октету $u=u_1 u_2\ldots u_8$ ставится в соответствие многочлен
$u(x)=u_1 x^7+u_2 x^6 +\ldots + u_8$. 
%
Многочлен ставится в соответствие также любому непустому
двоичному слову из целого числа октетов.
Как и при представлении слов числами используется 
соглашение <<от младших к старшим>>:
первому октету соответствует многочлен $u_1(x)$,
второму~--- $x^8 u_2(x)$, третьему~--- $x^{16}u_3(x)$ и так далее.

Многочлены $u(x)$ считаются многочленами над полем~$\FF_2$. 
Это значит, что при сложении и умножении многочленов операции над их
коэффициентами выполняются по модулю~$2$.
%
Деление $u(x)$ на ненулевой $f(x)$ состоит в определении многочленов 
$q(x)$, $r(x)$ таких, что $u(x)=q(x)f(x)+r(x)$ и степень $r(x)$ меньше 
степени $f(x)$. 
Многочлен $r(x)$ является остатком от деления.

Операция~$\ast$ состоит в умножении слов как многочленов с заменой результата
умножения на его остаток от деления на~$f(x)=x^{128}+x^7+x^2+x+1$. 
%
Выбранный многочлен~$f(x)$ является неприводимым (его нельзя представить
в виде произведения многочленов меньших степеней).
Поэтому операция~$\ast$ задает умножение слов как элементов поля из $2^{128}$
элементов (подробнее см.~\cite{LidNid88}).

\section{Запись перечислений}\label{DEFS.Seqs}

При записи последовательности~$u_1,u_2,\ldots,u_n$ 
допускается, если не оговорено противное, выполнение неравенства $n<2$.
При $n=0$ идет речь о пустой последовательности, 
а при $n=1$~--- об одноэлементной последовательности $u_1$.

Аналогичные соглашения распространяются на запись конкатенации нескольких
слов, суммы нескольких слагаемых, цикла.
%
Например,
\begin{itemize}
\item
слово $u_1\parallel u_2\parallel\ldots\parallel u_n$ является пустым при $n=0$ 
и состоит из единственного фрагмента~$u_1$ при $n=1$;
\item
сумма $u_1\oplus u_2\oplus\ldots\oplus u_n$ равняется $u_1$ при $n=1$;
%
% \doubt{и не определена при $n=0$}?
\item
тело цикла <<для $i=1,2,\ldots,n$ выполнить~\ldots>>
не выполняется ни разу, если $n=0$, и выполняется один раз, если~$n=1$.
\end{itemize}


