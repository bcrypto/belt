\section{Преобразование ключа}\label{KEYREP}

\subsection{Интерфейс}\label{KEYREP.IFace}

Преобразование ключа задается алгоритмом~$\algname{belt-keyrep}$.

Входными данными~$\algname{belt-keyrep}$ являются 
преобразуемый ключ~$X\in\{0,1\}^n$,
его уровень~$D\in\{0,1\}^{96}$,
заголовок~$I\in\{0,1\}^{128}$ нового ключа и его длина~$m$.
%
Должны соблюдаться ограничения: $m,n\in\{128,192,256\}$, $m\leq n$.
%
Выходными данными является преобразованный ключ~$Y\in\{0,1\}^m$.

При преобразовании ключа~$Y$ его уровень следует полагать 
равным~$D\boxplus\itob{1}_{96}$.

Используются алгоритмы~$\algname{belt-compress}$ и $\algname{belt-keyexpand}$,
определенные в~\ref{COMPR} и~\ref{KEYEXPAND}.

\subsection{Переменные}\label{KEYREP.Vars}

Используются переменные~$r\in\{0,1\}^{32}$ и~$s\in\{0,1\}^{256}$.

\subsection{Алгоритм преобразования ключа}\label{KEYREP.Alg}

Преобразование ключа~$\algname{belt-keyrep}(X,D,I,m)$
выполняется следующим образом:
\begin{enumerate}
\item
Присвоить переменной $r$ значение:
\begin{enumerate}
\item[1)]
$\hex{B194BAC8}$, если $n=m=128$;
\item[2)]
$\hex{5BE3D612}$, если $n=192$ и $m=128$;
\item[3)]
$\hex{5CB0C0FF}$, если $n=m=192$;
\item[4)]
$\hex{E12BDC1A}$, если $n=256$ и $m=128$;
\item[5)]
$\hex{C1AB7638}$, если $n=256$ и $m=192$;
\item[6)]
$\hex{F33C657B}$, если $n=m=256$.
\end{enumerate}

\item
Установить~$s\leftarrow\algname{belt-keyexpand}(X)$.

\item
Установить~$(\perp,s)\leftarrow\algname{belt-compress}
(r\parallel D\parallel I\parallel s)$. 

\item
Установить $Y\leftarrow\Lo(s, m)$.

\item
Возвратить $Y$.
\end{enumerate}
