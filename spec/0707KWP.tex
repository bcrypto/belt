\section{Аутентифицированное шифрование ключа}\label{KWP}

\subsection{Интерфейс}\label{KWP.IFace}

Аутентифицированное шифрование ключа задается алгоритмами установки 
защиты~$\algname{belt-kwp}$ и снятия защиты~$\algname{belt-kwp}^{-1}$.

Входными данными $\algname{belt-kwp}$ являются защищаемый 
ключ~$X\in\{0,1\}^{8*}$, его заголовок~$I\in\{0,1\}^{128}$ и ключ 
защиты~$K\in\{0,1\}^{256}$. Длина~$X$ должна быть не меньше~$128$.
%
Выходными данными является защищенный ключ~$Y\in\{0,1\}^{|X|+128}$.

Входными данными $\algname{belt-kwp}^{-1}$ являются защищенный
ключ~$Y\in\{0,1\}^*$, его заголовок~$I\in\{0,1\}^{128}$ и ключ
защиты~$K\in\{0,1\}^{256}$.
%
Выходными данными является либо признак ошибки~$\perp$,
либо исходный ключ~$X\in\{0,1\}^{|Y|-128}$.
%
Возврат~$\perp$ означает нарушение целостности входных данных.

Используются алгоритмы~$\algname{belt-wblock}$, $\algname{belt-wblock}^{-1}$,
определенные в~\ref{WBLOCK}.

\subsection{Переменные}\label{KWP.Vars}

При снятии защиты используется переменная~$r\in\{0,1\}^{128}$.

\subsection{Алгоритм установки защиты}\label{KWP.Wrap}

Установка защиты~$\algname{belt-kwp}(X,I,K)$ выполняется следующим образом:
\begin{enumerate}
\item
$Y\leftarrow\algname{belt-wblock}(X\parallel I,K)$.

\item
Возвратить~$Y$.
\end{enumerate}

\subsection{Алгоритм снятия защиты}\label{KWP.Unwrap}

Снятие защиты~$\algname{belt-kwp}^{-1}(Y,I,K)$ выполняется следующим образом:
\begin{enumerate}
\item
Если длина~$Y$ не кратна~$8$ или~$|Y|<256$, то возвратить~$\perp$.
\item
$(X\parallel r)\leftarrow \algname{belt-wblock}^{-1}(Y,K)$.
\item
Если $r\neq I$, то возвратить~$\perp$.
\item
Возвратить~$X$.
\end{enumerate}
