\section{Шифрование с сохранением формата}\label{FMT}

\subsection{Интерфейс}\label{FMT.IFace}

Шифрование с сохранением формата задается алгоритмами зашифрования
$\algname{belt-fmt}$ и расшифрования~$\algname{belt-fmt}^{-1}$.

Входными данными $\algname{belt-fmt}$ являются 
слово~$X\in\ZZ_m^n$, ключ~$K\in\{0,1\}^{256}$ 
и синхропосылка~$S\in\{0,1\}^{128}$.
%
Выходными данными является зашифрованное слово~$Y\in\ZZ_m^n$.

Входными данными $\algname{belt-fmt}^{-1}$ являются 
зашифрованное слово~$Y\in\ZZ_m^n$, ключ~$K\in\{0,1\}^{256}$ 
и синхропосылка~$S\in\{0,1\}^{128}$.
%
Выходными данными является расшифрованное слово~$Y\in\ZZ_m^n$.

Размер алфавита~$\ZZ_m$ и длина~$n$ входного и выходного слов 
должны принадлежать интервалу~$\{2,3,\ldots,65536\}$.

Используются алгоритмы~$\algname{belt-block}$ и~$\algname{belt-wblock}$,
определенные в~\ref{BLOCK} и~\ref{WBLOCK}.

\subsection{Подготовка входных данных}\label{FMT.Data}

Длина~$n$ записывается в виде суммы~$n_1+n_2$,
где~$n_1=\lceil n/2\rceil$, $n_2=\lfloor n/2\rfloor$.
%
По слагаемому~$n_i$, $i=1,2$, определяется минимальное натуральное~$b_i$ 
такое, что
\begin{enumerate}
\item[1)]
$b_i$ кратно~$64$;
\item[2)]
$2^{b_i}\geq m^{n_i}$.
\end{enumerate}

Синхропосылка~$S$ разбивается на фрагменты~$(S_1,S_2,S_3,S_4)=\Split(S,32)$. 
%
Дополнительно строятся слова~$S_0$ и~$S_5$, описывающие формат: 
$S_0=S_5=\itob{m}_{16}\parallel\itob{n}_{16}$. 

\subsection{Вспомогательные алгоритмы, переменные и обозначения}\label{FMT.Aux}

{\bf Алгоритм~\algname{belt-32block}}.
Используется алгоритм~\algname{belt-32block},
который принимает на вход слово~$t\in\{0,1\}^{192}$ и ключ~$K\in\{0,1\}^{256}$
и возвращает преобразованное слово~$t$. В слове~$t$ выделяются 
фрагменты $(t_1,t_2,t_3)=\Split(t,64)$.

Шаги алгоритма:
\begin{enumerate}
\item
Для $i=1,2,3$:
\begin{enumerate}
\item
$(t_2\parallel t_3)\leftarrow
\algname{belt-block}(t_2\parallel t_3,K)\oplus \itob{i}_{64}$; 
\item
$t\leftarrow t_2\parallel t_3\parallel(t_1\oplus t_2)$.
\end{enumerate}
\item
Возвратить~$t$.
\end{enumerate}

{\bf Переменная~$r$}.
Используется переменная~$r\in\ZZ_m^n$.
Переменная записывается в виде~$r=r_1\parallel r_2$,
$|r_i|=n_i$.

{\bf Переменные~$t_1$, $t_2$}.
Используются переменные~$t_1\in\{0,1\}^{64+b_1}$ 
и~$t_2\in\{0,1\}^{64+b_2}$.

{\bf Обозначение~\algname{str2bin}}.
Для~$j\in\{1,2\}$ и~$u\in\ZZ_m^*$ через~$\algname{str2bin}(u,b_j)$ 
обозначается двоичное слово~$\itob{\btoi{u}_m}_{b_j}$.

{\bf Обозначение~\algname{bin2str}}.
Для~$j\in\{1,2\}$ и~$t\in\{0,1\}^{64*}$ через~$\algname{bin2str}(t,n_j)$ 
обозначается слово~$\itob{\btoi{t}}_{m,n_j}$. 

{\bf Обозначение~\algname{roundf}}.
Пусть $t\in\{0,1\}^{64*}$, причем~$|t|\geq 128$.
Обозначение~$\algname{roundf}(t,K)$ раскрывается как:
\begin{enumerate}
\item[1)]
$\algname{belt-block}(t,K)$ при $|t|=128$,
\item[2)]
$\algname{belt-32block}(t,K)$ при $|t|=192$ и
\item[3)]
$\algname{belt-wblock}(t,K)$ при $|t|\geq 256$.
\end{enumerate}

\subsection{Константы}\label{FMT.Const}

В алгоритмах используются константы $C_0,C_1,\ldots,C_5\in\{0,1\}^{32}$.
Константы определяются по~первым двум строкам таблицы~\ref{Table.SBOX}
и имеют следующий вид:
\begin{align*}
&
C_0 = \hex{B194BAC8},\quad
C_1 = \hex{0A08F53B},\quad
C_2 = \hex{366D008E},\quad
C_3 = \hex{584A5DE4},\\
&
C_4 = \hex{8504FA9D},\quad
C_5 = \hex{1BB6C7AC}.
\end{align*}

\subsection{Алгоритм зашифрования}

Зашифрование~$\algname{belt-fmt}(X,K,S)$ выполняется следующим образом:
\begin{enumerate}
\item
Установить $r\leftarrow X$.
\item
Для $i=1,2,3$:
\begin{enumerate}
\item
$t_2\leftarrow
\algname{roundf}(\algname{str2bin}(r_2,b_2)\parallel C_{2i-2}\parallel S_{2i-2},K)$;
\item
$r_1\leftarrow r_1\oplus\algname{bin2str}(t_2,n_1)$;
\item
$t_1\leftarrow 
\algname{roundf}(\algname{str2bin}(r_1,b_1)\parallel C_{2i-1}\parallel S_{2i-1},K)$; 
\item
$r_2\leftarrow r_2\oplus\algname{bin2str}(t_1,n_2)$.
\end{enumerate}
\item
Установить $Y\leftarrow r_1\parallel r_2$.
\item
Возвратить $Y$.
\end{enumerate}

\subsection{Алгоритм расшифрования}

Расшифрование~$\algname{belt-fmt}^{-1}(Y,K,S)$ выполняется следующим образом:
\begin{enumerate}
\item
$r\leftarrow Y$.
\item
Для $i=3,2,1$:
\begin{enumerate}
\item
$t_1\leftarrow 
\algname{roundf}(\algname{str2bin}(r_1,b_1)\parallel C_{2i-1}\parallel S_{2i-1},K)$;
\item
$r_2\leftarrow r_2\ominus \algname{bin2str}(t_1,n_2)$;
\item
$t_2\leftarrow 
\algname{roundf}(\algname{str2bin}(r_2,b_2)\parallel C_{2i-2}\parallel S_{2i-2},K)$;
\item
$r_1\leftarrow r_1\ominus \algname{bin2str}(t_2,n_1)$.
\end{enumerate}
\item
Установить~$X\leftarrow r_1\parallel r_2$.
\item
Возвратить $X$.
\end{enumerate}

