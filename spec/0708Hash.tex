\section{Хэширование}\label{HASH}

\subsection{Интерфейс}\label{HASH.IFace}

Хэширование задается алгоритмом~$\algname{belt-hash}$.

Входными данными~$\algname{belt-hash}$ является сообщение~$X\in\{0,1\}^*$.
%
Выходными данными является хэш-значение~$Y\in\{0,1\}^{256}$.

Используется алгоритм~$\algname{belt-compress}$, определенный в~\ref{COMPR}.

\subsection{Переменные}\label{HASH.Vars}

Используются переменные~$r,s,t\in\{0,1\}^{128}$ и~$h\in\{0,1\}^{256}$.

\subsection{Алгоритм хэширования}\label{HASH.Alg}

Хэширование~$\algname{belt-hash}(X)$ выполняется следующим образом:
\begin{enumerate}
\item
Определить $(X_1,X_2,\ldots,X_n)=\Splita(X, 256)$.

\item
Установить
\begin{enumerate}
\item
$r\leftarrow \itob{|X|}_{128}$;
\item
$s\leftarrow 0^{128}$;
\item
$h\leftarrow\hex{B194BAC80A08F53B366D008E584A5DE48504FA9D1BB6C7AC252E72C202FDCE0D}$ 
(см. первые две строки таблицы~\ref{Table.SBOX});
\item
$X_n\leftarrow X_n \parallel 0^{256-|X_n|}$.
\end{enumerate}

\item
Для $i=1,2,\ldots,n$ выполнить:
\begin{enumerate}
\item
$(t,h)\leftarrow\algname{belt-compress}(X_i\parallel h)$;
\item
$s\leftarrow s\oplus t$.
\end{enumerate}

\item
Установить 
$(\perp,Y)\leftarrow\algname{belt-compress}(r\parallel s\parallel h)$.

\item
Возвратить $Y$.
\end{enumerate}

\begin{note*}
Если $X=\perp$, то $n=1$ и $X_1=\perp$.
\end{note*}
