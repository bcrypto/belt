\section{Шифрование в режиме гаммирования с обратной связью}\label{CFB}

\subsection{Интерфейс}\label{CFB.IFace}

Шифрование в режиме гаммирования с обратной связью задается алгоритмами 
зашифрования~$\algname{belt-cfb}$ и расшифрования~$\algname{belt-cfb}^{-1}$.

Входными данными $\algname{belt-cfb}$ являются сообщение~$X\in\{0,1\}^*$, 
ключ~$K\in\{0,1\}^{256}$ и синхропосылка~$S\in\{0,1\}^{128}$.
%
Выходными данными является зашифрованное сообщение~$Y\in\{0,1\}^{|X|}$.

Входными данными $\algname{belt-cfb}^{-1}$ являются зашифрованное 
сообщение~$Y\in\{0,1\}^*$, ключ~$K\in\{0,1\}^{256}$ и 
синхропосылка~$S\in\{0,1\}^{128}$. 
%
Выходными данными является расшифрованное сообщение~$X\in\{0,1\}^{|Y|}$.

Используется алгоритм~$\algname{belt-block}$, определенный в~\ref{BLOCK}.

\subsection{Алгоритм зашифрования}\label{CFB.Encr}

Зашифрование~$\algname{belt-cfb}(X,K,S)$ выполняется следующим образом: 
\begin{enumerate}
\item
Определить~$(X_1,X_2,\ldots,X_n)=\Split(X, 128)$.
\item
Обозначить~$Y_0=S$.
\item
Для $i=1,2,\ldots,n$ выполнить:
$Y_i\leftarrow X_i\oplus\Lo(\algname{belt-block}(Y_{i-1},K), |X_i|)$.
\item
Возвратить
$Y=Y_1\parallel Y_2\parallel\ldots\parallel Y_n$.
\end{enumerate}

\subsection{Алгоритм расшифрования}\label{CFB.Decr}

Расшифрование~$\algname{belt-cfb}^{-1}(Y,K,S)$ выполняется следующим образом: 
\begin{enumerate}
\item
Определить~$(Y_1,Y_2,\ldots,Y_n)=\Split(Y, 128)$.
\item
Обозначить~$Y_0=S$.
\item
Для $i=1,2,\ldots,n$ выполнить: 
$X_i\leftarrow X_i\oplus\Lo(\algname{belt-block}(Y_{i-1},K), |Y_i|)$.
\item
Возвратить
$X=X_1\parallel X_2\parallel\ldots\parallel X_n$.
\end{enumerate}
