\section{Шифрование в режиме счетчика}\label{CTR}

\subsection{Интерфейс}\label{CTR.IFace}

Шифрование в режиме счетчика задается алгоритмами 
зашифрования~$\algname{belt-ctr}$ и расшифрования~$\algname{belt-ctr}^{-1}$.

Входными данными $\algname{belt-ctr}$ являются сообщение~$X\in\{0,1\}^*$, 
ключ~$K\in\{0,1\}^{256}$ и синхропосылка~$S\in\{0,1\}^{128}$.
%
Выходными данными является зашифрованное сообщение~$Y\in\{0,1\}^{|X|}$.

Входными данными $\algname{belt-ctr}^{-1}$ являются зашифрованное 
сообщение~$Y\in\{0,1\}^*$, ключ~$K\in\{0,1\}^{256}$ и 
синхропосылка~$S\in\{0,1\}^{128}$. 
%
Выходными данными является расшифрованное сообщение~$X\in\{0,1\}^{|Y|}$.

Используется алгоритм~$\algname{belt-block}$, определенный в~\ref{BLOCK}.

\subsection{Переменные}\label{CTR.Vars}

Используется переменная~$s\in\{0,1\}^{128}$.

\subsection{Алгоритм зашифрования}\label{CTR.Encr}

Зашифрование~$\algname{belt-ctr}(X,K,S)$ выполняется следующим образом:
\begin{enumerate}
\item
Определить $(X_1,X_2,\ldots,X_n)=\Split(X,128)$.
\item
Установить 
$s\leftarrow \algname{belt-block}(S,K)$.
\item
Для $i=1,2,\ldots,n$ выполнить:
\begin{enumerate}
\item
$s\leftarrow s\boxplus\itob{1}_{128}$,
\item
$Y_i\leftarrow X_i\oplus\Lo(\algname{belt-block}(s,K),|X_i|)$.
\end{enumerate}
\item
Возвратить~$Y=Y_1\parallel Y_2\parallel\ldots\parallel Y_n$.
\end{enumerate}

\subsection{Алгоритм расшифрования}\label{CTR.Decr}

Расшифрование~$\algname{belt-ctr}^{-1}(Y,K,S)$ выполняется следующим образом:
\begin{enumerate}
\item
Определить $(Y_1,Y_2,\ldots,Y_n)=\Split(Y,128)$.
\item
Установить 
$s\leftarrow \algname{belt-block}(S,K)$.
\item
Для $i=1,2,\ldots,n$ выполнить:
\begin{enumerate}
\item
$s\leftarrow s\boxplus\itob{1}_{128}$,
\item
$X_i\leftarrow Y_i\oplus\Lo(\algname{belt-block}(s,K),|Y_i|)$.
\end{enumerate}
\item
Возвратить~$X=X_1\parallel X_2\parallel\ldots\parallel X_n$.
\end{enumerate}
